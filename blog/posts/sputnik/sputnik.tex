Durante las Navidades aproveché para ver la serie Chernobyl. Una de las escenas que más me impresionó es la que sucede durante el juicio, donde el físico nuclear Valery Legasov realiza una explicación magistral sobre las causas de la explosión del reactor 4. Hay un momento en el que Legasov explica que el reactor 4 de Chernobyl estaba diseñado para funcionar a 3200 MW (3200 millones de watios) y que en su pico, justo antes de la explosión, superó la friolera de 33,000 MW. ¿Eso qué es? ¿Mucho? 

Para contextualizar, el 28 de diciembre de 2020 se alcanzó el máximo historico en generación eólica al conseguir 19,588 MW. La demanda de energía eléctrica (instantanea) en la península no suele sobrepasar los 32,000 MW (https://demanda.ree.es/visiona/peninsula/demanda/total/2020-12-31) por lo que tanto en ese pico el reactor 4 habría abastecido a todo un país en pleno 2020.

Al hablar de potencia consumida me viene a la cabeza ipso facto el Sputnik 1. Este invento, al igual que los reactores nucleares de tipo RMBK, también era soviético y fue el primer satélite en orbitar la Tierra y transmitir una señal de radio desde el espacio consumiendo tan sólo 1 vatio de potencia (un cargador de móvil estándar consume unos 5 W. 


El Sputnik tenía una forma esférica de dimensiones no muy superiores a las de un pelota de playa, un peso de 83.6 kg y 4 antenas que colgaban de la semiesfera frontal. Como cualquier satélite incorporaba un subsistema de generación de potencia, un transmisor radio que operaba a 2 frecuencias (20.005 y 40.002 MHz) y un sistema de control térmico formado por un ventilador que se activaba al superar los 36 grados.

% sputnik_1

El subsistema de generación de potencia estaba formado por 3 baterías: 2 alimentaban al transmisor de radio y la restante al sistema de regulación de temperatura. Estas baterías de plata-zinc estaban diseñadas para aguantar 2 semanas. Sin embargo sobrepasaron las expectativas permitiendo transmitir la señal de radio durante 22 días.

Con el Sputnik, los soviéticos pretendían conseguir una seríe de objetivos científicos como modelar de manera más precisa la densidad de las capas bajas de la atmósfera a partir del deterioro orbital _1_ o estudiar el impacto de los micrometeoritos a partir de la presión interior del satélite _2_. Sin embargo, los soviéticos querían apuntarse el tanto y demostrar su supremacia tecnológica... por segunda vez. El primer golpe ya lo habían dado en el campo nuclear en 1953 con el desarrollo de la primera bomba atómica. 

Lyndon Johnson, expresidente de los EEUU, dijo la frase que posiblemente exprese mejor las dimensiones del carrera espacial:

``El Imperio Romano controló el mundo porqué pudo construir una red de caminos y carreteras. Después, cuando nos empezamos a desplazar por mar, fue el Imperio Británico el dominante al disponer de barcos. En la era aérea, nosotros fuimos los más dominantes porqué tuvimos aviones. Ahora los Comunistas han dejado su huella en el espacio exterior. Los Estados Unidos se encuentran en una carrera por la supervivencia''

_1_
(reducción de la altura del satélite debido a la fricción producida por la atmósfera)

_2_
El interior del satélite estaba presurizado con nitrogeno a 1.3 atm (131 kPa). Si la presión interior caía por debajo de 34.3 kPa se activaría un interruptor barométrico cambiando la longitud de onda de la señal enviada [2].

El 4 de octubre de 1957 cualquier americano podía con un receptor de radio y una antena sintonizar la frecuencia de 20 MHz y oir el particular ``beep'' que emitía el satélite durante uno de sus pasos visibles por la Tierra.

http://mentallandscape.com/Sputnik1_WashingtonDC.mp3

Intentando empatizar con un americano del momento... los soviéticos habían puesto encima de sus cabezas un aparato que por el momento tan sólo emitía un tono agudo y monótono, pero que nadie sabía ni lo que pretendía ni lo que contenía, ni lo que duraría. La URSS también había demostrado la capacidad para poner una cabeza nuclear sobrevolando un estado enemigo. Los americanos se sentían impotentes ante la conquista del espacio del enemigo soviético. 


Consecuencias
Estados Unidos intentó reaccionar lo antes posible y así demostrar su fortaleza y sus avances en cuanto a tecnología espacial y mísiles. Ésto, les precipitó al lanzamiento del Vanguard TV-3 que supuso un fiasco estrepitoso. El 6 de diciembre de 1957 el cohete Vanguard explotaba dos segundos después de alcanzar una altitud de 2 metros.

Sin embargo, meses después, el 17 de marzo de 1958 Estados Unidos conseguía poner el Vanguard 1, su primer satélite artificial, en órbita.

- International Geophysical Year 
	https://www.nesdis.noaa.gov/content/1957–58-year-satellite
- NASA
- Guerra fria
- Korolev


http://mentallandscape.com/S_Sputnik1.htm
[2] http://www.russianspaceweb.com/sputnik_design.html