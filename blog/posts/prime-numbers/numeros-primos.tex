13/08/2019

\begin{document}
Números Primos

No recuerdo la primera vez que escuche el concepto de número primo. Debería ser durante la educación primaria, una vez aprendidas las tablas de multiplicar, cuando algún profesor me inculcó ese concepto sin aparente importancia. Los números primos pertenecen al grupo de conceptos que todo el mundo conoce, o al menos ha escuchado, pero que suele quedar en el olvido si no se mantiene contacto. Por ello, cabe recordar que un número primo es aquel que únicamente es divisible por él mismo y la unidad (e.g. 13 sería primo ya que sólo se puede dividir entre 13 y 1). 

La primera vez que tuve una conciencia acerca de los números primos fue en secundaria con la descomposición factorial de los números. En este proceso descompones cualquier número en una multiplicación de números primos:

\begin{equation}
	15 = 3 x 5 \\ [.3em]
	18 = 2 x 3 x 3
\end{equation}

Recuerdo las típicas reglas para saber rápidamente si un número era o no divisible por 2, 3, 5, ....

\begin{itemize}
	\item Divisible entre 2. Simplemente si el número es par será divisible por\footnote{mal uso de la traducción by del inglés. Preferiblemente entre} 2
	\item Divisible entre 3. Si la suma de las cifras da un número divisible entre 3 (e.g. 225 --> 2+2+5=9).
	\item Divisible entre 5. Si acaba en 0 o 5.
\end{itemize}

A partir del 7, la tarea se complica. Ya no es tan obvio ver si un número es divisible entre 13 o entre 19. Para estos casos se realizaba una simple división y si el resto era 0 significaba que era divisible. Más tarde veremos que Carl Gauss desarrolló toda la matemática modular que, entre otras cosas, nos permite obtener el resto de una división.

- Hermanos
- Criba de Eratostenes
- Fermat
- Euler & Mersenne
- Gauss vs Legendre
- Riemann
- Dirichlet
- Hardy y Littlewood


Aplicaciones
- RSA


\end{document}